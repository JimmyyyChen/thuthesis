% !TeX root = ../thuthesis-example.tex

% 中英文摘要和关键字

\begin{abstract}
  本研究提出了一种交互式软件开发环境bolt.SE,旨在将大语言模型(LLM)的代码生成能力与现代软件工程最佳实践深度融合。该系统基于开源项目bolt.diy构建,通过四个核心创新模块增强了LLM辅助开发的工程化程度:API优先开发模块基于OpenAPI规范实现API定义与管理,使LLM能够理解并生成符合接口规范的代码;测试驱动开发(TDD)模块将"测试先行"理念与代码生成相结合,通过结构化测试约束引导LLM生成高质量代码;模型上下文协议(MCP)模块提供标准化接口使LLM能够安全调用外部工具和数据源,突破知识局限;持续集成与部署模块则结合了TDD模块与MCP模块,实现了从测试定义到自动部署的全链路自动化。bolt.SE有效解决了当前LLM辅助开发面临的上下文理解受限、反馈机制不完善及工程化流程缺失等挑战,为构建更加高效、易用的软件开发环境提供了新思路。
  \thusetup{
  keywords = {大语言模型, 软件工程, 测试驱动开发, API优先开发, 模型上下文协议},
  }
\end{abstract}

\begin{abstract}
  This study introduces an interactive software development environment, bolt.SE, designed to deeply integrate the code generation capabilities of large language models (LLMs) with modern software engineering best practices. Built upon the open-source project bolt.diy, the system enhances the engineering rigor of LLM-assisted development through four core innovation modules: The API-first development module leverages the OpenAPI specification to define and manage APIs, enabling LLMs to generate code compliant with interface standards; the test-driven development (TDD) module incorporates the "test-first" approach, guiding LLMs to produce high-quality code through structured testing constraints; the Model Context Protocol (MCP) module provides standardized interfaces that allow LLMs to safely interact with external tools and data sources, thereby overcoming knowledge limitations; and the continuous integration and deployment module integrates the TDD and MCP modules to achieve end-to-end automation from test definition to deployment. bolt.SE effectively addresses key challenges faced by current LLM-assisted development, such as limited context understanding, insufficient feedback mechanisms, and lack of engineering workflows, offering new ideas for building more efficient and user-friendly development environments.
  \thusetup{
    keywords = {large language models, software engineering, test-driven development, API-first development, model context protocol},
  }
\end{abstract}
