% !TeX root = ../thuthesis-example.tex

\chapter{结论与展望}
\label{chap:conclusion}

本研究深入探索大型语言模型(LLM)与现代软件工程最佳实践的协同融合,通过构建bolt.SE平台实现了从自然语言需求到可运行软件的自动化生成流程。在此过程中,系统性地解决了当前LLM辅助开发面临的上下文理解有限、反馈机制不完善及工程化流程缺失等核心挑战。本章总结研究成果,分析关键贡献,并展望未来发展方向。

\section{主要研究成果}

本研究的核心技术创新主要体现在以下四个方面:

\begin{enumerate}
  \item API优先开发模块:构建了基于OpenAPI规范的API定义与管理系统,使LLM能够理解、调用及生成符合API规范的代码。通过结构化API描述,系统实现了外部服务与LLM的无缝对接,扩展模型功能边界,减少上下文占用,提升多API协同效率。用户可通过直观界面添加、编辑API定义,系统支持API密钥、Bearer令牌等多种认证方式。
  
  \item 测试驱动开发(TDD)模块:将测试先行理念与LLM代码生成深度结合,通过结构化测试定义引导模型生成满足预定义行为的代码实现,建立了基于"红-绿-重构"循环的开发验证机制。用户在生成代码前先定义期望行为,系统将测试约束转化为LLM可理解的指导信息,确保生成代码符合需求并具备可验证性。
  
  \item 模型上下文协议(MCP)模块:引入标准化接口协议,使LLM能够安全调用外部工具和数据源。系统通过模块化架构设计实现了多传输方式(stdio、HTTP SSE)和多服务类型的统一管理,支持从本地文件访问到远程服务调用的广泛场景。MCP不仅突破了LLM知识截止限制,还显著降低了输出幻觉风险,为复杂问题解决提供了交互式支持。
  
  \item CI/CD自动化集成:通过MCP与TDD协同,实现了从测试定义到代码生成再到自动部署的全链路自动化流程。系统可以利用标准化工具接口(如GitLab API)自动完成仓库创建、配置生成、代码推送与流水线触发等操作,形成"测试—代码—仓库—流水线—部署"的完整开发闭环,有效简化了传统DevOps流程中的手动操作环节。
\end{enumerate}

这些技术创新共同构建了一个功能完备、流程闭环的软件开发环境,展现了LLM在规范化软件工程约束下的潜力。

本研究通过多个实例验证了bolt.SE的实用价值, 这些应用案例验证了各模块的独立功能,为不同背景用户提供了从构思到部署的端到端解决方案。

\section{与bolt.diy开源项目的连接}

本研究与开源项目bolt.diy保持着密切的联系,形成了一个相互促进、持续迭代的创新循环。bolt.SE以bolt.diy为基础开发,继承了其在WebContainer技术、多LLM支持和对话式代码生成等方面的核心优势,并在此基础上进行了深入的工程化拓展,尤其是在软件工程方法(如TDD、API-first、MCP集成)方面进行了创新与完善。

与此同时,本研究主动参与bolt.diy的社区建设,向其代码库提交了具体的贡献,包括引入MCP模块以及修复多个已知的缺陷,这些贡献已通过PR提交至社区(\href{https://github.com/stackblitz-labs/bolt.diy/pull/1704}{https://github.com/stackblitz-labs/bolt.diy/pull/1704}),处于社区讨论与审阅阶段。此外,本研究设计与实现的TDD模块、APIActions模块与MCP集成功能也将陆续以独立组件的形式回馈到bolt.diy项目中,以开源的方式提供给更广泛的开发者群体,推动整个生态系统的健康成长。

\section{未来发展方向}

bolt.SE 在编程教育领域具备广阔的应用潜力。系统能够作为个性化学习平台,帮助初学者以自然语言描述生成可运行代码并获得即时反馈,从而有效降低编程入门的门槛。内置的 TDD 与 API-First 等模块为软件工程课程提供了实践基础,使学生能够在真实开发环境中理解并应用现代开发方法,提升规范化开发能力。此外,bolt.SE 让教师实时了解学生的开发过程并给予有针对性的指导,同时也为课程项目提供了原型快速验证的能力,让学生能够专注于创新和设计。未来,系统计划与清华大学的课程结合,推动平台在实际教学中的应用与落地,结合课程内容提供场景化的编程辅助。

作为核心连接机制,MCP 拥有广阔的扩展空间。系统将进一步支持图像、音频等多模态开发工具,覆盖如 UI 设计和多媒体应用等复杂场景,并逐步集成面向医疗、金融、教育等领域的专用工具包,以满足行业化开发需求。基于 MCP 的分布式协同机制有助于促进团队间资源与进度的实时共享,支持跨地域的协同开发。系统还将持续优化工具选择算法,使其能够根据任务上下文自适应组合所需功能,进一步简化用户操作流程。随着 MCP 社区的持续发展,MCP 生态正逐步走向标准化与模块化,推动开放工具市场的建设,实现多类专业功能在 LLM 应用开发流程中的高效集成。

\section{总结}

bolt.SE将大语言模型的自然语言处理能力与软件工程的规范化实践深度融合,创建了一个从需求描述到可运行软件的端到端解决方案。通过TDD、API-First、MCP和CI/CD四大模块的协同工作,系统有效解决了当前LLM辅助开发面临的核心挑战,为构建更高效、更易用的软件开发环境提供了新思路。

未来bolt.SE将持续探索LLM在软件工程全生命周期中的深度应用,融入到实际使用场景,并通过与bolt.diy等开源项目的持续协作,推动整个智能软件开发生态系统的发展。

随着LLM技术与软件工程方法论的不断融合,像bolt.SE这样的系统将成为连接人类创意与代码实现的重要桥梁,使软件开发更加自然、高效且具有创造性。 