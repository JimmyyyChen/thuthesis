% !TeX root = ../thuthesis-example.tex

\chapter{引言}

\section{选题背景}

近年来,大语言模型(Large Language Models, LLMs)在人工智能领域取得了突破性进展,显著推动了软件开发技术与方法论的革新。这些模型具备强大的自然语言处理和代码生成能力,能够将用户描述的需求快速而准确地转化为代码,从而极大降低了软件开发领域的技术进入壁垒。特别是以GitHub Copilot为代表的LLM辅助开发工具,已展现出令人瞩目的代码自动生成与辅助编程性能,在提升开发效率和代码生产力方面取得了积极的初步成果。然而,尽管当前LLM工具在特定编程任务中表现突出,仍然存在诸多技术瓶颈制约其更广泛和更深入的应用。例如,这些工具往往缺乏对复杂代码仓库与运行环境上下文的有效感知与理解,进而难以实现与既有开发环境的无缝衔接;同时,由于缺乏完善的实时代码执行及在线预览机制,开发者难以即时验证生成代码的准确性和适用性;此外,这类工具普遍缺乏对完整软件工程流程的系统性支持,如体系架构设计、自动化测试、持续集成与部署等关键环节。因此,探索如何有效融合LLMs的先进特性与软件工程的最佳实践,建立系统化的集成工具链,以应对上述挑战,已成为当前软件工程领域的重要研究课题。

\section{问题提出}

尽管已有的LLM辅助编程工具在实际应用中初步证明了其潜力,但仍然存在多个关键挑战亟待解决。首先,在处理大型与复杂的软件代码库时,目前的LLM工具普遍面临对代码上下文环境理解不足的问题。这种不足具体体现在对于代码模块间接口、依赖关系、命名规范、编码标准以及系统整体架构设计约定等信息的识别与理解不够充分,从而导致生成的代码难以与既有环境实现高效融合,降低了生成代码的实际应用价值与长期可维护性。其次,现有工具缺乏完整、成熟的网页端实时代码运行及预览能力,使开发人员难以实时直观地观察和验证生成代码的具体表现及效果,极大限制了开发过程的敏捷性与交互性。此外,目前LLM工具的设计尚未有效整合广泛接受的现代软件工程实践,导致开发流程缺乏系统化和自动化的管理机制,进而削弱了代码质量保障和开发过程掌控的有效性。

例如,API优先开发(API-First)、测试驱动开发(TDD)、数据库自动化配置,以及持续集成与持续交付(CI/CD)等先进实践,均是本研究所重点关注的软件工程流程和最佳实践。

基于以上现实问题,本研究提出了一种面向无代码基础用户与编程初学者的交互式软件开发环境——bolt.SE。其中,“bolt”来源于开源项目bolt.diy,bolt.diy是一款基于Web技术构建的开源开发平台,旨在提供全栈开发环境,并支持大语言模型集成,体现了该平台的技术基础,“SE”则表示软件工程(Software Engineering),突出该环境对软件工程领域方法论的深度融合与系统化支持。该环境旨在将大语言模型的自然语言处理与代码生成优势,与现代软件工程领域公认的最佳实践深度融合,进而实现更加高效且易用的LLM驱动的软件原型开发工具。具体而言,bolt.SE将在上下文理解机制、代码实时执行与预览功能、软件工程方法论的自动化与系统化支持等方面进行重点突破,力图有效解决现存LLM工具的上述挑战,最终显著提高软件开发的生产效率与代码质量。

\section{本文结构}

TODO