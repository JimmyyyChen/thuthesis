% !TeX root = ../thuthesis-example.tex

\chapter{引言与研究背景}

\section{研究背景与意义}
近年来,大语言模型(Large Language Models, LLMs)在代码生成领域展现出显著进步,成为自然语言与可执行代码之间桥梁能力提升的重要驱动力。相关系统性综述表明,LLM 在 HumanEval、MBPP 等主流基准任务中的表现持续提升,反映出模型在理解与生成能力上的重大突破\cite{jiang2024survey}。同时,学界也在不断完善对代码生成准确性、效率及可读性的评测方法,推动评测体系向更高标准发展\cite{chen2024eval}。在工业界,AI 编程辅助工具的应用快速普及。GitHub 的调研数据显示,92\% 的美国开发者已在实际开发中采用此类工具\cite{githubblog2023};《金融时报》报道,SWE-bench 评测中 LLM 的问题解决率已从 2023 年的 4.4\% 显著提升至 69.1\%,进一步印证了 AI 驱动开发效率的提升\cite{ft2025ai}。

尽管 LLM 相关研究和应用持续取得突破,如何将其能力真正融入大型软件工程实践,仍面临诸多挑战,包括上下文理解受限、反馈机制不完善以及工程化流程缺失等。因此,将 LLM 的生成能力与软件工程最佳实践紧密结合,如测试驱动开发(Test-Driven Development, TDD)、API 优先(API-First)开发和持续集成/持续部署(Continuous Integration/Continuous Deployment, CI/CD),已成为当前亟待解决的重要课题。

\section{当前 LLM 辅助开发工具发展与挑战}

为系统梳理主流 LLM 编程工具在实际开发环境中的差异,表~\ref{tab:llm-tools-updated} 汇总了几款代表性产品在集成平台、主要功能及反馈机制方面的对比。当前具有影响力的典型工具包括 GitHub Copilot X、Cursor、v0.dev、Replit、Lovable、Windsurf 和 bolt.new 等。


\begin{table}[htbp]
  \centering
  \footnotesize
  \caption{典型 LLM 辅助开发工具功能对比}
  \label{tab:llm-tools-updated}
  \setlength{\tabcolsep}{3pt}
  \begin{tabular}{
    >{\raggedright\arraybackslash}p{2.4cm}
    >{\raggedright\arraybackslash}p{2.6cm}
    >{\raggedright\arraybackslash}p{3.6cm}
    >{\raggedright\arraybackslash}p{4.6cm}
  }
    \toprule
    \textbf{工具名称} & \textbf{集成平台} & \textbf{主要功能} & \textbf{反馈机制} \\
    \midrule
    GitHub Copilot\cite{copilotx2023}      & VS Code/JetBrains 扩展        & 多模型(GPT-4o、Claude 3.5、Gemini 1.5 Pro)行内补全、代码审查支持      & 无内置 CI 流水线反馈                                \\
    \addlinespace
    Cursor\cite{cursor2025wiki}            & 独立 AI IDE(VS Code 分支)    & 对话式多文件生成、跨文件重构、Diff 视图              & 实时代码差异预览;无原生 TDD/CI/CD 集成            \\
    \addlinespace
    v0.dev\cite{v0dev2024}                 & 浏览器内 Chat UI               & 基于 React/Tailwind 的 UI 自动生成                  & 页面预览与即时部署;缺少测试与流水线支持            \\
    \addlinespace
    Replit\cite{replit2025wiki}            & 云端 IDE                       & Ghostwriter AI、Agent V2 对话式生成、协同编辑       & 实时代码运行、部署管道;无结构化测试反馈            \\
    \addlinespace
    Lovable\cite{lovable2025times}         & Web App Builder                & 对话式前后端+数据库一体化生成                      & 即时预览与调试;无原生 CI/CD 流水线                \\
    \addlinespace
    Windsurf\cite{windsurf2024}            & AI 增强 IDE                    & 行内补全、多文件编辑、聊天问答                      & 变更高亮与上下文聊天;尚无 TDD/CI 插件            \\
    \addlinespace
    bolt.new\cite{arunachalam2024boltnew,boltnewgithub2024} & 浏览器 IDE(StackBlitz WebContainers) & 从提示到部署:项目依赖管理、Node.js 运行、前后端一体化生成 & 终端输出、即时预览与部署状态反馈;支持文件系统与第三方 API 调用 \\
    \bottomrule
  \end{tabular}
\end{table}

GitHub Copilot X 自 2023 年起,在多模型切换和多文件补全方面取得了显著进展,集成了 GPT-4o、Claude 3.5 和 Gemini 1.5 Pro,支持基于项目全局上下文的代码建议生成,并新增了代码审查和安全提示等能力,但与 CI/CD 或 TDD 流程的深度集成仍有待加强\cite{copilotx2023}。Cursor 通过"vibe coding"工作流,将对话式多文件生成与实时 Diff 视图结合,提升了多文件协同开发效率。然而,其自动化测试和持续交付方案尚不完整,部分流程仍需借助外部工具补充\cite{cursor2025wiki}。v0.dev 主打对话生成 React/Tailwind 前端代码和一键部署,适合原型设计及 UI 快速验证,但缺乏对测试的原生支持\cite{v0dev2024}。Replit 借助 Ghostwriter AI 和 Agent V2,实现了云端协作开发与即时运行体验,但结构化测试和 CI/CD 功能仍主要依赖外部插件\cite{replit2025wiki}。Lovable 能在单轮对话中生成全栈前后端与数据库代码,支持即时预览,但缺乏完善的版本控制与自动化流水线,难以满足团队长期协作需求\cite{lovable2025times}。Windsurf 将行内补全与上下文对话结合,提升代码建议的语义丰富性,但暂未实现与 TDD 或 CI/CD 的深度耦合\cite{windsurf2024}。bolt.new 依托 StackBlitz WebContainers,提供浏览器内完整的依赖安装、Node.js 服务与代码生成能力,终端输出和部署状态反馈为快速验证提供便利,但在测试驱动和持续交付方面仍有扩展空间\cite{arunachalam2024boltnew,boltnewgithub2024}。

可以看到,尽管上述工具在各自领域取得了一定进展,但在应用于实际软件工程实践中仍存在几项关键挑战:

\subsection{上下文理解局限}

在处理大型与复杂软件代码库时,当前工具面临对代码上下文理解不足的问题。GitHub Copilot 等早期工具主要基于当前打开的文件内容进行补全,对项目全局信息缺乏深入理解\cite{DivaCopilot}。即使是 Cursor 和 Windsurf 等支持多文件编辑的工具,在理解代码模块间接口关系、命名规范以及整体架构设计约定等方面仍有不足,导致生成的代码难以与既有环境高效融合。

\subsection{交互反馈与即时验证不足}

现有工具普遍缺乏完整、成熟的网页端实时代码运行及预览能力,使开发人员难以即时观察和验证生成代码的实际表现。虽然 bolt.new 通过 WebContainers 提供了浏览器内运行环境,v0.dev 和 Lovable 支持 UI 代码即时预览,但它们在复杂项目的测试验证方面仍不完善。多数平台尚未实现生成代码的即时测试反馈,依赖与配置问题难以及时被发现和解决,开发流程中从代码生成到测试验证的环节也未能做到无缝衔接,用户体验仍然割裂。

\subsection{软件工程实践融合不足}

当前 LLM 辅助工具尚未有效整合现代软件工程的最佳实践。尽管 Replit 提供了部分部署管道,但对于测试驱动开发、API 优先设计以及持续集成与部署等核心流程的支持仍显不足。这种融合上的欠缺,导致工具在多个方面存在局限。首先,缺乏基于测试定义指导代码生成的系统化方法,使开发过程难以保证质量可控。其次,API 开发过程中规范定义与实现往往相互割裂,难以确保一致性和可维护性。最后,从代码生成到部署的全链路自动化程度较低,仍需开发者投入大量人工操作,难以实现高效闭环。这些问题共同凸显了一个系统化软件工程环境的必要性,即不仅要充分发挥 LLM 在代码生成方面的能力,还应将其与软件工程最佳实践紧密结合,形成覆盖需求定义、代码验证到系统部署的完整开发流程。

\section{bolt.SE的目标}

基于以上分析,本研究提出了面向无代码基础用户与编程初学者的交互式软件开发环境——bolt.SE。其中,"bolt"来源于开源项目bolt.diy,体现该平台的技术基础;"SE"则表示软件工程(Software Engineering),突出该环境对软件工程方法论的深度融合与系统化支持。

bolt.SE旨在将大语言模型的自然语言处理与代码生成优势,与现代软件工程领域公认的最佳实践深度融合,实现更加高效且易用的LLM驱动软件原型开发工具。具体目标包括:

\begin{itemize}
    \item 系统引入API优先开发模块,增强LLM与外部服务的交互能力。通过基于OpenAPI规范的API定义与管理,让LLM理解并生成符合API规范的代码,实现外部服务与LLM的无缝对接,有效扩展模型功能边界。
    \item 通过结构化的测试驱动开发模块,改善代码生成质量与可靠性。该模块让用户能够在生成代码前先定义期望行为,将测试约束转化为LLM可理解的指导信息,建立基于"红-绿-重构"循环的验证机制,确保生成代码符合预定义行为。
    \item 通过模型上下文协议(MCP)模块,突破LLM知识与功能边界。此模块提供标准化接口,支持LLM安全调用外部工具和数据源,实现访问远程服务调用的多场景支持。
    \item 通过MCP与TDD协同实现CI/CD自动化流程,完成从测试定义到部署的全链路自动化。系统利用标准化工具接口自动完成仓库创建、配置生成、代码推送与流水线触发等操作,形成完整开发闭环,简化传统手动操作环节。
\end{itemize}

这些目标共同构成一个面向无代码基础用户的端到端软件开发环境,使开发者能够专注于创意表达与业务逻辑,而非繁琐的编程细节,提高软件原型设计与开发的可达性与效率。

本研究将基于bolt.diy开源框架构建bolt.SE平台,实现对API优先开发、测试驱动开发与模型上下文协议等核心技术模块的集成,以达到前述目标。

\section{论文结构}
论文结构安排如下:

第1章(本章)介绍研究背景、问题提出与目标,分析当前LLM辅助开发工具的发展与挑战,概述bolt.SE的核心设计。

第\ref{chap:code-generation}章详细介绍bolt.SE系统的整体架构、核心技术及创新点。该章首先介绍bolt.diy的基础架构与技术特点,包括开源与扩展性、多种大语言模型支持、代码生成能力以及WebContainer技术等方面。随后重点阐述bolt.SE的创新模块及其在软件工程流程中的应用价值,包括API优先开发模块、测试驱动开发模块和模型上下文协议模块。

第\ref{chap:tdd}章详细阐述测试驱动开发模块的概念与技术实现,分析TDD在LLM驱动开发中的价值,并通过具体案例展示测试定义如何引导LLM生成高质量代码。该章将介绍bolt.SE测试功能的技术实现,并通过JavaScript计算器的开发流程,展示基于"红–绿–重构"循环的测试驱动开发过程。

第\ref{chap:api-first}章聚焦API优先开发模块,介绍OpenAPI规范在bolt.SE中的应用,API定义与管理机制,以及LLM如何理解并调用API实现功能扩展。该章将分析bolt.SE中APIActions的实现方案,展示结构化API定义如何帮助LLM突破知识局限,通过与外部服务交互扩展其能力边界。

第\ref{chap:mcp}章深入剖析模型上下文协在bolt.SE中的原理与实现,分析MCP如何通过标准化工具接口使LLM能够安全地访问外部资源,突破知识截止限制,降低输出幻觉风险,并支持实时数据获取与专业工具调用。通过IoTDB数据库集成案例,展示MCP在统一自然语言交互中同时发现并调用多类工具的强大能力。

第\ref{chap:mcp-tdd-cicd}章综合前述核心技术,展示MCP与TDD协同驱动的GitLab CI/CD自动化流程,体现bolt.SE在全链路软件工程实践中的系统集成能力。该章通过应用实例,详述从Jest测试定义到GitLab仓库创建、CI配置自动生成及部署的完整开发闭环,实现从"测试—代码—仓库—流水线—部署"的全链路自动化。

第\ref{chap:conclusion}章总结全文研究成果,阐述bolt.SE与bolt.diy开源项目的紧密联系和相互促进的创新循环,以及对社区的贡献。同时展望bolt.SE的应用潜力与未来计划。