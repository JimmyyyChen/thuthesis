% !TeX root = ../thuthesis-example.tex

\chapter{引言与研究背景}

\section{研究背景与意义}

近年来,大语言模型(Large Language Models, LLMs)在人工智能领域取得了突破性进展,显著推动了软件开发技术与方法论的革新。OpenAI 的 Codex 模型(Chen 等人,2021~\cite{Chen2021})和 GPT-3.5/4 系列(OpenAI,2022--2023~\cite{OpenAI2022})展示了从自然语言描述自动生成代码的能力,引发了软件开发范式变革的讨论~\cite{arxiv2401}。2022 年以来,学术界和工业界涌现出许多专注于代码的 LLM,如 Google 的 AlphaCode(Li 等人,2022~\cite{Li2022AlphaCode})、Meta 的 CodeLlama(Roziere 等人,2023~\cite{Roziere2023})以及开源社区的 StarCoder(Li 等人,2023~\cite{Li2023StarCoder})等~\cite{arxiv2401}。

这些模型具备强大的自然语言处理和代码生成能力,能够将用户描述的需求快速而准确地转化为代码,从而极大降低了软件开发领域的技术进入壁垒。特别是以GitHub Copilot为代表的LLM辅助开发工具,已展现出令人瞩目的代码自动生成与辅助编程性能,在提升开发效率和代码生产力方面取得了积极的初步成果。在函数级别的编程挑战(如 LeetCode、HumanEval)上,这些模型已经能达到甚至超越人类平均水平~\cite{LLMCodeGraph2024}。

然而,将 LLM 真正融入软件工程实践,仍面临着上下文理解、交互反馈和流程集成等多方面挑战。尽管当前LLM工具在特定编程任务中表现突出,这些技术瓶颈制约了其更广泛和更深入的应用。因此,探索如何有效融合LLMs的先进特性与软件工程的最佳实践,建立系统化的集成工具链,已成为当前软件工程领域的重要研究课题。

\section{当前LLM辅助开发工具分析}

为弥合 LLM 生成代码与实际开发环境之间的鸿沟,近年出现了一批交互式 LLM 辅助开发工具。这些工具通常集成在开发者常用的 IDE 或编辑器中,提供从代码建议、自动修改到运行结果预览的闭环体验。表~\ref{tab:llm-tools}对比了部分典型LLM辅助开发工具在集成平台、主要功能和反馈机制方面的特性。

\begin{table}[htbp]
\centering
\footnotesize
\caption{典型 LLM 辅助开发工具功能对比}
\label{tab:llm-tools}
\setlength{\tabcolsep}{3pt}
\begin{tabular}{>{\raggedright\arraybackslash}p{2.4cm}>{\raggedright\arraybackslash}p{2.6cm}>{\raggedright\arraybackslash}p{3.6cm}>{\raggedright\arraybackslash}p{4.6cm}}
\toprule
\textbf{工具名称} & \textbf{集成平台} & \textbf{主要功能} & \textbf{反馈机制} \\
\midrule
GitHub Copilot (2021) & VS Code 等 IDE 扩展 & 基于当前文件上下文提供单文件代码补全及行内生成建议 & 无实时运行反馈;对全局项目结构理解不足,初学者较难把握整体架构 \\
\addlinespace
Cursor (2023) & 基于 VS Code 衍生的专用编辑器 & 支持对话式代码修改、跨文件重构及差异展示 & 通过对话嵌入反馈修改建议;缺乏成熟的 CI/CD 与 TDD 支持 \\
\addlinespace
ChatGPT Code Interpreter (2023) & ChatGPT Web 界面 & 对话生成代码与 Python 沙盒执行,支持文件上传与结果展示 & 提供执行输出和文件交互反馈,但运行环境局限于 Python \\
\addlinespace
bolt.new (2024) & 浏览器内 WebContainers & 实现全流程全栈应用生成,支持依赖安装与即时部署 & 交互式反馈良好,用户体验稳定;作为商业产品具备高成熟度 \\
\addlinespace
bolt.diy (2024) & 浏览器内 WebContainers(开源) & 快速原型生成、低门槛入门,支持多 LLM 模型接入 & 开源优势明显,便于社区扩展;在工程实践支持上尚不完善 \\
\bottomrule
\end{tabular}
\end{table}

\subsection{主流工具发展概述}

\textbf{GitHub Copilot (2021)}作为早期商用的AI辅助编程工具,以VS Code扩展形式出现。它基于OpenAI Codex模型,在用户编码时给出行内自动补全建议。Copilot的架构较为简单:将当前文件上下文和光标位置的内容发送给LLM,返回可能的续写代码~\cite{DivaCopilot}。尽管提升了编码效率,其仅依赖当前文件上下文进行代码补全,对初学者而言理解整个项目结构存在难度;同时,缺乏交互式反馈机制与自动调试功能。

\textbf{Cursor (2023)}是基于VS Code开发的AI代码编辑器,支持行内补全、对话式代码修改以及跨文件重构~\cite{CursorReview}。该工具通过diff展示修改建议,使用户可直观了解代码变动。虽然融合了编辑、对话和执行等功能,但对初学者使用门槛较高,且在软件工程流程方面仍处于实验性探索阶段。

\textbf{ChatGPT Code Interpreter (2023)}实质上是ChatGPT的Python沙盒扩展,允许用户在对话中生成、执行代码,并上传数据文件获取运行结果~\cite{FirstBlush2023}。这种"所见即所得"的运行反馈降低了模型幻觉风险。然而,其运行环境主要局限于Python,对复杂项目支持不足,同时未能整合完整的软件工程流程。

\textbf{bolt.new和bolt.diy (2024)}为交互式代码生成领域带来新突破。bolt.new作为商业产品,依托StackBlitz的WebContainers技术,实现从提示、运行、编辑到部署的全流程开发体验;bolt.diy则在继承核心功能的同时,以开源模式降低使用门槛,支持多种LLM集成~\cite{BoltNew2024, BoltDIY2024}。尽管在简化开发环境搭建方面具优势,但在支持现代软件工程实践方面仍存在局限。

\subsection{发展趋势与技术挑战}

新一代工具正逐步引入实时验证与反馈环节,使LLM生成代码能立即在环境中检验,不断改进输出质量。这种人机协作的实时交互为构建\emph{bolt.SE}这类智能开发环境提供了技术基础。然而,现有工具仍面临三方面主要挑战:

1. 上下文理解受限:大多数工具缺乏对大型代码库的整体理解能力,难以有效掌握跨文件依赖关系和项目架构
2. 反馈机制不完善:实时验证与即时预览功能尚不成熟,阻碍了快速迭代与问题定位
3. 软件工程实践支持不足:缺乏对TDD、API-First、CI/CD等先进软件工程方法论的系统支持

这些挑战共同构成了当前LLM辅助开发工具的瓶颈,限制了其在实际软件工程中的应用价值。

\section{本研究的核心问题与创新点}

尽管已有的LLM辅助编程工具在实际应用中初步证明了其潜力,但仍然存在多个关键挑战亟待解决:

\subsection{上下文理解挑战}

在处理大型与复杂软件代码库时,目前LLM工具面临对代码上下文理解不足的问题。早期工具(如GitHub Copilot)只能利用当前打开文件内容进行补全,对整个项目全局信息缺乏了解~\cite{DivaCopilot}。研究表明,LLM在独立代码片段任务表现出色,但在仓库级任务上表现不佳~\cite{LLMCodeGraph2024}。

这种局限具体体现在对代码模块间接口、依赖关系、命名规范、编码标准以及系统整体架构设计约定等信息的理解不足,导致生成代码难以与既有环境实现高效融合,降低了实际应用价值与长期可维护性。

研究者已探索多种解决方案:一方面,扩大模型上下文窗口,如DeepSeek-Coder通过填空式训练策略扩展上下文至16K tokens~\cite{arxiv2401,Guo2024};另一方面,利用检索增强方法提供相关代码片段和设计信息,如CodeXGraph框架将代码库解析为图数据库,帮助LLM理解代码结构关系~\cite{LLMCodeGraph2024}。尽管有所进展,完全架构感知的代码生成仍是开放问题。

\subsection{交互反馈与即时验证挑战}

现有工具普遍缺乏完整、成熟的网页端实时代码运行及预览能力,使开发人员难以即时观察和验证生成代码的具体表现,限制了开发过程的敏捷性与交互性。这一问题导致:

1. 生成代码与实际运行环境存在差距,无法立即验证可行性
2. 依赖与配置问题难以及时发现与解决
3. 用户体验割裂,缺乏从代码生成到运行的无缝衔接

ChatGPT Code Interpreter通过Python沙盒提供了初步解决方案,而bolt.new利用WebContainers技术实现了更完整的运行环境。然而,这些工具仍缺乏与完整软件工程流程的整合,难以支持复杂项目的全生命周期管理。

\subsection{软件工程实践融合挑战}

LLM工具设计尚未有效整合现代软件工程实践,导致开发流程缺乏系统化和自动化管理机制,削弱了代码质量保障和开发过程掌控的有效性。

\textbf{测试驱动开发(TDD)}对LLM具有特殊价值。Mathews和Nagappan(2024)研究表明,提供测试用例能显著提高LLM生成代码的正确率~\cite{Mathews2024TDD},测试用例明确了预期行为,减少了模糊理解,推动了"生成-执行-验证"循环在AI辅助编程中的应用。

\textbf{持续集成与交付(CI/CD)}方面,研究显示ChatGPT等模型已用于代码审查和CI配置生成~\cite{ChatGPTUsage2023},但因训练数据时效性问题,可能生成过时配置,显示出LLM与CI/CD深度融合的挑战。

\textbf{API优先开发(API-First)}领域,研究采用分而治之方法处理大型API规范~\cite{DivaCopilot},如先让LLM阅读API说明并输出实现计划,再逐条接口生成代码。Ericsson公司实验表明,这种方法能产出相当比例正确样板代码,但仍需人工审查修改。

\section{bolt.SE的创新设计}

基于以上分析,本研究提出了面向无代码基础用户与编程初学者的交互式软件开发环境——bolt.SE。其中,"bolt"来源于开源项目bolt.diy,体现该平台的技术基础;"SE"则表示软件工程(Software Engineering),突出该环境对软件工程方法论的深度融合与系统化支持。

bolt.SE旨在将大语言模型的自然语言处理与代码生成优势,与现代软件工程领域公认的最佳实践深度融合,实现更加高效且易用的LLM驱动软件原型开发工具。具体创新包括:

\subsection{系统化软件工程流程支持}

bolt.SE将重点集成三项核心软件工程实践:

1. \textbf{测试驱动开发(TDD)模块}:提供从测试定义到代码验证的一体化支持,使LLM生成的代码始终受到测试规范的严格约束
   
2. \textbf{API优先开发(API-First)模块}:基于OpenAPI规范的API定义、管理与调用机制,支持LLM理解并生成符合API规范的代码

3. \textbf{模型上下文协议(MCP)}:提供标准化接口,使LLM能够安全地调用外部工具和数据源,扩展功能边界

这三大模块协同工作,形成完整的开发闭环,如\ref{chap:mcp-tdd-cicd}章所示的"测试定义—功能实现—代码验证—版本控制—自动部署"全链路。

\subsection{交互式开发与反馈机制}

bolt.SE在bolt.diy基础上强化了即时反馈与交互验证能力:

1. 实时代码执行与预览,支持多语言环境
2. 测试结果可视化与错误定位
3. API调用验证与响应展示
4. 工具链调用可视化与执行状态追踪

这些功能使开发者能够即时验证LLM生成代码的正确性,并通过反馈循环不断优化结果。

\subsection{上下文理解与工具协作增强}

针对上下文理解挑战,bolt.SE提供:

1. 结构化API与测试定义,为LLM提供精确约束
2. MCP工具集成,使LLM能访问外部数据源与工具
3. 代码仓库理解与多文件协同修改支持

这些机制使bolt.SE能够更好地理解并适应复杂软件项目的上下文环境。

\section{研究方法与论文结构}

本研究采用系统设计与实验验证相结合的方法,通过bolt.SE平台实现对核心技术模块的集成与评估。论文结构安排如下:

第一章(本章)介绍研究背景、问题提出与创新点,分析当前LLM辅助开发工具的发展与挑战,概述bolt.SE的核心设计。

第\ref{chap:tdd}章详细阐述测试驱动开发(TDD)模块的概念与技术实现,分析TDD在LLM驱动开发中的价值,并通过具体案例展示测试定义如何引导LLM生成高质量代码。

第\ref{chap:api-first}章聚焦API优先开发模块,介绍OpenAPI规范在bolt.SE中的应用,API定义与管理机制,以及LLM如何理解并调用API实现功能扩展。

第\ref{chap:mcp}章深入剖析模型上下文协议(MCP)的原理与实现,MCP如何为LLM提供标准化工具接口,并通过案例展示MCP在实际开发中的应用。

第\ref{chap:mcp-tdd-cicd}章综合前述核心技术,展示MCP与TDD协同驱动的GitLab CI/CD自动化流程,体现bolt.SE在全链路软件工程实践中的系统集成能力。

通过这一结构安排,本研究将系统阐述bolt.SE如何通过技术创新与软件工程实践融合,解决当前LLM辅助开发面临的核心挑战,为构建更高效、更易用的软件开发环境提供新思路。